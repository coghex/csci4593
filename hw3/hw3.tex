\documentclass[12pt, notitlepage, final]{article} 

\newcommand{\name}{Vince Coghlan}

\usepackage{amsfonts}
\usepackage{amssymb}
\usepackage{amsmath}
\usepackage{latexsym}
\usepackage{enumerate}
\usepackage{amsthm}
\usepackage{nccmath}
\usepackage{setspace}
\usepackage[pdftex]{graphicx}
\usepackage{epstopdf}
\usepackage[siunitx]{circuitikz}
\usepackage{tikz}
\usepackage{float}
\usepackage{cancel} 
\usepackage{setspace}
\usepackage{overpic}
\usepackage{mathtools}
\usepackage{listings}
\usepackage{color}

\numberwithin{equation}{section}
\DeclareRobustCommand{\beginProtected}[1]{\begin{#1}}
\DeclareRobustCommand{\endProtected}[1]{\end{#1}}
\newcommand{\dbr}[1]{d_{\mbox{#1BR}}}
\newtheorem{lemma}{Lemma}
\newtheorem*{corollary}{Corollary}
\newtheorem{theorem}{Theorem}
\newtheorem{proposition}{Proposition}
\theoremstyle{definition}
\newtheorem{define}{Definition}
\newcommand{\column}[2]{
\left( \begin{array}{ccc}
#1 \\
#2
\end{array} \right)}

\newdimen\digitwidth
\settowidth\digitwidth{0}
\def~{\hspace{\digitwidth}}

\setlength{\parskip}{1pc}
\setlength{\parindent}{0pt}
\setlength{\topmargin}{-3pc}
\setlength{\textheight}{9.0in}
\setlength{\oddsidemargin}{0pc}
\setlength{\evensidemargin}{0pc}
\setlength{\textwidth}{6.5in}
\newcommand{\answer}[1]{\newpage\noindent\framebox{\vbox{{\bf CSCI 4593 Spring 2014} 
\hfill {\bf \name} \vspace{-1cm}
\begin{center}{Homework \#3}\end{center} } }\bigskip }

%absolute value code
\DeclarePairedDelimiter\abs{\lvert}{\rvert}%
\DeclarePairedDelimiter\norm{\lVert}{\rVert}
\makeatletter
\let\oldabs\abs
\def\abs{\@ifstar{\oldabs}{\oldabs*}}
%
\let\oldnorm\norm
\def\norm{\@ifstar{\oldnorm}{\oldnorm*}}
\makeatother

\def\dbar{{\mathchar'26\mkern-12mu d}}
\def \Frac{\displaystyle\frac}
\def \Sum{\displaystyle\sum}
\def \Int{\displaystyle\int}
\def \Prod{\displaystyle\prod}
\def \P[x]{\Frac{\partial}{\partial x}}
\def \D[x]{\Frac{d}{dx}}
\newcommand{\PD}[2]{\frac{\partial#1}{\partial#2}}
\newcommand{\PF}[1]{\frac{\partial}{\partial#1}}
\newcommand{\DD}[2]{\frac{d#1}{d#2}}
\newcommand{\DF}[1]{\frac{d}{d#1}}
\newcommand{\fix}[2]{\left(#1\right)_#2}
\newcommand{\ket}[1]{|#1\rangle}
\newcommand{\bra}[1]{\langle#1|}
\newcommand{\braket}[2]{\langle #1 | #2 \rangle}
\newcommand{\bopk}[3]{\langle #1 | #2 | #3 \rangle}
\newcommand{\Choose}[2]{\displaystyle {#1 \choose #2}}
\newcommand{\proj}[1]{\ket{#1}\bra{#1}}
\def\del{\vec{\nabla}}
\newcommand{\avg}[1]{\langle#1\rangle}
\newcommand{\piecewise}[4]{\left\{\beginProtected{array}{rl}#1&:#2\\#3&:#4\endProtected{array}\right.}
\newcommand{\systeme}[2]{\left\{\beginProtected{array}{rl}#1\\#2\endProtected{array}\right.}
\def \KE{K\!E}
\def\Godel{G$\ddot{\mbox{o}}$del}

\lstset{language=C}
\onehalfspacing

\begin{document}

\answer{}

\textbf{3.1)} I just did it half-byte by half-byte. $D-A=3,\;etc...$
\begin{center}
  \begin{tabular}{ccccc}
  &5&E&D&4\\
  -&0&7&A&4\\
  \hline
  &5&E&3&0\\
  \end{tabular}
\end{center}


\textbf{3.4)} This was done in a similar fasion, except that I needed to carry a
one in from the fourth octal number.
\begin{center}
  \begin{tabular}{ccccc}
    &$\cancel{4}^3$&3&6&5\\
  -&3&4&1&2\\
  \hline
  &0&7&5&3\\
  \end{tabular}
\end{center}


\textbf{3.6)} 63.  Neither over or underflow, since since this number fits well within an 8 bit value.

\textbf{3.12)} $62 \times 12 = 744$, lets see if we can do this using the process that a computer uses.
\begin{center}
  \begin{tabular}{| c | l | c | c | c |}
    \hline
    Iteration & Step & Multiplier & Multiplicant & Product \\
    \hline
    \hline
    0 & Initial Values & 1100 & 0000111110 & 0000000000 \\
    \hline
    1 & 0 $\Rightarrow$ Nop & 1100 & 0000111110 & 0000000000 \\
      & Shift left m-cand & 1100 & 0001111100 & 0000000000 \\
      & Shift right m-plier & 0110 & 0001111100 & 0000000000 \\
    \hline
    2 & 0 $\Rightarrow$ Nop & 0110 & 0001111100 & 0000000000 \\
      & Shift left m-cand & 0110 & 0011111000 & 0000000000 \\
      & Shift right m-plier & 0011 & 0011111000 & 0000000000 \\
    \hline
    3 & 1 $\Rightarrow$ add to product & 0011 & 0011111000 & 0011111000 \\
      & Shift left m-cand & 0011 & 0111110000 & 0011111000 \\
      & Shift right m-plier & 0001 & 0111110000 & 0011111000 \\
    \hline
    4 & 1 $\Rightarrow$ add to product & 0001 & 0111110000 & 1011101000 \\
      & Shift left m-cand & 0001 & 1111100000 & 1011101000 \\
      & Shift right m-plier & 0000 & 1111100000 & 1011101000 \\
    \hline

  \end{tabular}
\end{center}
Since we are out of multiplier, we can see that the product is indeed 744.

\textbf{3.17)} We can represent $\text{0x}33 \times \text{0x}55$ as $(2\times2\times2\times2\times2\times2 + \text{0x}15)\times\text{0x}33$ or
\[
  (\text{0x}33 << 6) + (2\times2\times2\times2 + \text{0x}5)\times\text{0x}33
\]
\[
  (\text{0x}33 << 6) + (\text{0x}33 << 4) + (\text{0x}33 << 2) + \text{0x}33
\]
\[
  = 0x10EF \text{ or } 4335
\]

\textbf{3.19)}
\begin{center}
  \begin{tabular}{| c | l | c | c | c |}
    \hline
    Iteration & Step & Quotient & Divisor & Remainder \\
    \hline
    \hline
    0 & Initial Values & 0000 & 1000100000 & 00111100 \\
    \hline
    1 & rem=rem-div & 0000 & 1000100000 & \text{negative} \\
      & rem$<$0$\Rightarrow$+Div, sll Q, $Q_0=0$ & 0000 & 1000100000 & 00111100 \\
      & shift div right & 0000 & 0100010000 & 00111100 \\
    \hline
    2 & rem=rem-div & 0000 & 0100010000 & \text{negative} \\
      & rem$<$0$\Rightarrow$+Div, sll Q, $Q_0=0$ & 0000 & 0100010000 & 00111100 \\
      & shift div right & 0000 & 0010001000 & 00111100 \\
    \hline
    3 & rem=rem-div & 0000 & 0010001000 & \text{negative} \\
      & rem$<$0$\Rightarrow$+Div, sll Q, $Q_0=0$ & 0000 & 0010001000 & 00111100 \\
      & shift div right & 0000 & 0001000100 & 00111100 \\
    \hline
    4 & rem=rem-div & 0000 & 0001000100 & \text{negative} \\
      & rem$<$0$\Rightarrow$+Div, sll Q, $Q_0=0$ & 0000 & 0001000100 & 00111100 \\
      & shift div right & 0000 & 0000100010 & 00111100 \\
    \hline
    5 & rem=rem-div & 0000 & 0000100010 & 00011010\\
      & rem$\geq 0 \Rightarrow$ sll Q, $Q_0=1$ & 0001 & 0000100010 & 00011010 \\
      & shift div right & 0001 & 0000010001 & 00011010 \\
    \hline
    6 & rem=rem-div & 0001 & 0000010001 & 00001001\\
      & rem$\geq 0 \Rightarrow$ sll Q, $Q_0=1$ & 0011 & 0000010001 & 00001001 \\
      & shift div right & 0011 & 0000001000 & 00001001 \\
    \hline
  \end{tabular}
\end{center}

And our answer is $q = 3$ and $r = 11$.  Lo and behold, if we pull out a calculator and
type in $3*21 + 11$ we will attain 74.

\textbf{3.22)} 0x0C000000 Gives us that $s=0$, $e=24$, and $m=1$ ($1$.fraction). This will
be:
\[
  (-1)^0\times 1\times 2^{24-127} \approx 9.8607613\cdot10^{-32}
\]

\textbf{3.23)} First I will figure out the fraction like so:
\[
  63.25 = 2^5+2^4+2^3+2^2+2^1+2^0+2^{-2}
\]
Which is going to look like: 111111.01.  Our mantissa bit will therefore take the form
111110100000...  We know that the exponent will need to be the largest bit, $2^5$ so we
can know that the exponent field is 5+127 or 10000100.  The sign bit is obviously zero.
Our number is therefor: 01000010011111010000000000000000.  It is easier to read 0x427d0000.


\end{document}
